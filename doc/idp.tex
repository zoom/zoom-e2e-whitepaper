\textbf{Note:} Identity Provider Attestations are not currently available, and planned for a future
update.

Accounts that have an ADN and an identity provider\footnote{Compatible identity providers will need
to support a custom extension of the OpenID Connect (OIDC) protocol which we describe below.} (\idp)
will be able to have the \idp vouch for the identity of their users in a way that other Zoom users
can independently verify. This mechanism restricts the ability, even for Zoom insiders, to
impersonate account members. Many organizations already trust an \idp for authentication purposes,
so this feature does not increase the attack surface or require additional trust in the \idp.

In order for clients to be able to verify identity attestations by an external \idp, we need two
components:

\begin{enumerate}
\item A way for clients to determine the \idp associated with a Zoom account (that cannot be
    tampered with by the Zoom servers)
\item A mechanism for {\idp}s to issue---and for clients to verify---a signed attestation that binds
    a user's email address to their set of devices and keys
\end{enumerate}

\subsection{Associating Accounts with Identity Providers}
\label{subsec:adnToIdp}
In order to associate an account to its \idp, we will rely on TLS\@. Accounts with ADNs will host a
JSON object at the root of the subdomain \texttt{idp-config.well-known.} of the ADN, e.g.,
\texttt{https://idp-config.well-known.example.org}.

The JSON object contains a field with key \texttt{us.zoom.idp.<CLOUD\_NAME>} whose value points to
the unique domain that the \idp has reserved for this account. For example, if \texttt{example.org}
is using \texttt{generic-idp.com} as their identity provider and has an account hosted on the Zoom
commercial cloud, the object might look like:

\begin{Verbatim}
{
    "us.zoom.idp.commercial": "examplecorp.generic-idp.com"
}
\end{Verbatim}

Specifying the cloud in the field name allows to support ADNs configuring different IDPs for
different clouds. Note that since accounts are expected to change their \idp rarely, clients can
cache this mapping aggressively.

If the endpoint does not return a valid response, clients will assume there is no \idp and not
display any identifiers for this user. Because Zoom will proxy client connections to ADNs to avoid
leaking the IP addresses of Zoom clients to external parties, a malicious Zoom server (or network
attackers) can convince clients that a certain account is not using an \idp by e.g., returning a
timeout error or a DNS NXDOMAIN message. We believe that this is the right tradeoff for the
following reasons:

\begin{enumerate}
\item Configuring the \idp via subdomains offers flexibility for the account admins. For example,
    the account admin can delegate the \texttt{idp-config.well-known} subdomain to the \idp, which
    can obtain a TLS certificate for the subdomain and host the required JSON file. The only
    technical action required by the account admin would be adding a DNS record for the subdomain.
\item Once deployed, our transparency layer (Section~\ref{sec:ztt}) will ensure that if the server
    tampers with an account's ADN or \idp, the misbehavior will be detected through auditing of the
    account and ADN sigchains.
\end{enumerate}

\subsection{IDP Attestations}

\idp attestations will be generated and verified according to the OpenID Connect (OIDC) protocol.
OIDC is an extension of the widely used OAuth 2.0 authentication protocol, an industry standard that
many {\idp}s and Single Sign-On providers already support. OIDC provides a standardized format, the
ID token, to express claims about identities and their attributes. It also specifies how users can
request attestations for their own identity and verify ones obtained from other users.

We will customize the protocol by

\begin{enumerate}
\item Introducing an additional attribute \texttt{"zoom-identity-snapshot"} to the ID token in order
    to encode the state of a user's identity on Zoom. {\idp}s will keep track of the latest value of
    this attribute for every account user.
\item Specifying how this attribute can only be updated by the authorized user, and not by any other
    user or entity, including Zoom servers.
\item Specifying how ID tokens about a specific user identity can be validated by other users.
\end{enumerate}

Our modified OIDC ID token (which will also be referred to as an \idp attestation) is a signed JSON
Web Token (JWT) data structure which contains claims about a user's Zoom identity. The payload might
look like the following:

\begin{Verbatim}
{
  "iss": "https://examplecorp.generic-idp.com", // issuer

  "name": "Alice Henderson", // optional
  "email": "alice@example.com", // optional

  "zoom-identity-snapshot": "409788…",

  "exp": 1311281970 // expiration time
  "iat": 1311280970, // issue time

  [...]
}
\end{Verbatim}

The token contains an issuer field which identifies the OIDC issuer of the token (the \idp). In
order for Zoom clients to accept an identity claim, the issuer field must match the domain that
associates the \idp with the account (in our example, as returned by
\texttt{https://example.org/.well-known/idp-config}). \texttt{name} and \texttt{email} are marked as
optional to allow users to certify that they are part of a specific account without disclosing
exactly which member they are. \texttt{iat} and \texttt{exp} define the validity of the token.

The \texttt{zoom-identity-snapshot} field encodes the state of a user's identity. As explained in
Section~\ref{subsec:sigchains}, clients keep track of these states using sigchains. An identity
snapshot will encode the user sigchain tail (which itself commits to the \userID); however, the
identity provider can treat this attribute as an opaque string and does not need to check its
validity.

\subsection{Updating Snapshots}
Users who successfully sign into Zoom using OAuth 2.0 receive an access token from the \idp which
can be used to access protected resources. This access token will also be used to read and update
their own \texttt{zoom-identity-snapshot} attribute. To ensure that only an authorized user on a
Zoom native client is able to update the identity snapshot stored by the \idp, the Identity Snapshot
protocol requires {\idp}s to:

\begin{enumerate}
\item Introduce a new OAuth scope \texttt{update-zoom-identity} which is required to update the
    snapshot.
\item Only issue access tokens with the \texttt{update-zoom-identity} scope for requests that use
    PKCE and have set the redirect URI to a fixed custom URI intended to refer to the native Zoom
    desktop and mobile applications: \texttt{us.zoom://idp\_auth}.
\end{enumerate}

With the custom URI redirect, we trust the operating system and browser to redirect to the native
Zoom app and not to a website in a browser: such a website might be serving malicious JavaScript
from a compromised web server that could hijack the authorization flow. PKCE is an OAuth 2.0
extension that prevents other apps installed on the user's device from intercepting the
authorization code. The Zoom app will not share the resulting ``write" access token with anyone
else, including the Zoom server, but read-only access to snapshots can be extended to all access
tokens, including those issued to browser sessions.

We realize that the protections given to write ID tokens depend on the security of the underlying
platform including the user's browser, their OS and their hardware, but we intend these protections
to be best effort measures.

Whenever devices make updates to their user sigchain, they submit the latest sigchain tail to their
\idp. In the event where a device successfully makes a sigchain update with the Zoom server but
fails to update the \idp, the next device that comes online will notice the new sigchain links and
update the \idp itself, but only after the user has reviewed the new links for potentially malicious
device additions.

\subsection{Validating IDP Attestations}

\idp attestations can be validated like standard OIDC ID tokens in the Authorization Code
Flow~\cite{oidc} with a few modifications. Users will:

\begin{enumerate}
\item Verify that the \idp in the \texttt{iss} field of the JWT matches the JSON hosted at the ADN's
    \texttt{idp-config.well-known.} subdomain. This ensures that the IDP is authorized to sign on
    behalf of the account ADN (as specified in Section~\ref{subsec:adnToIdp}).
\item Use OpenID Connect Discovery (e.g., make a request to \\
    \texttt{https://examplecorp.generic-idp.com/.well-known/openid-configuration}) \\ to ensure that
    the key used to sign the JWT is valid.
\item Validate the JWT, including checking its signature and expiration date.
\item Validate that the email and snapshot match the user sigchain provided by the Zoom server. When
    used in E2EE meetings, users will accept attestations that do not cover the latest sigchain tail
    as long as the new links since the {\idp}'s snapshot do not revoke the device currently being
    used in the meeting. If the new links change the user's email or account, then the user will be
    shown without identifiers.
\end{enumerate}

Note that the fetched attestation may be shared with and validated by anyone, so it doesn't include
the \texttt{aud} field.

\subsection{Security Properties}

Users in accounts with an \idp that supports our protocol receive particularly strong security
guarantees. Because clients rely on the account's ADN (and not the Zoom server) to determine the
\idp, a Zoom insider cannot impersonate these users unless the \idp or TLS itself is compromised.
The Zoom proxy or network attackers can potentially trick a user into believing that another user's
account's ADN does not have an associated \idp, but in this case, the user will appear without an
email address or ADN.
