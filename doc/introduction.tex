\section{Introduction}
Millions of people use Zoom to learn, connect with friends and family, collaborate with colleagues,
and deliver vital services. Zoom users deserve excellent security, and Zoom is working to provide
these protections in a transparent and peer-reviewed process. This document describes the
cryptographic protocols powering several Zoom products and features, both upcoming and already
released.

We are actively engaging in a process of consultation with multiple stakeholders, including clients,
cryptography experts, and civil society. As we receive feedback, we will update this document to
reflect changes in roadmap and cryptographic design.

\subsection{Outline}

In Section~\ref{sec:background_and_goals}, we start by providing some context on the Zoom platform
and discussing the goals and limitations of our approach.

Many of our most secure offerings, including end-to-end encryption (E2EE), require Zoom client
devices to have cryptographic keys whose secret components are not available to Zoom servers.
Section~\ref{sec:identitykeymanagement} describes these keys, how they are managed by Zoom clients,
and how we use \textit{sigchains} to bind keys from a user's devices to that user's account and
identifiers.

In Section~\ref{sec:ztt}, we propose a key transparency architecture that will force Zoom servers to
be consistent about each user's identity, empowering Zoom client devices to monitor their own
identities and detect any attempts at impersonation. In Section~\ref{sec:idpattestations}, we
describe how users can leverage external identity providers to certify their own keys, allowing
communication partners to independently verify them without relying on Zoom. Both of these
mechanisms reduce the need for fingerprints and security codes to achieve authentication.

The following sections introduce the cryptographic protocols powering Zoom Mail Service
(Section~\ref{sec:email}), Meetings (Section~\ref{sec:meetings}) and Phone
(Section~\ref{sec:phone}).

\subsubsection{E2E Encryption for Zoom Meetings}
Before version\footnote{Previous versions available at
\url{https://github.com/zoom/zoom-e2e-whitepaper/}} 4.0, this document was titled ``E2E Encryption
for Zoom Meetings,'' as it focused primarily on the Meetings product, for which it proposed an
incrementally deployable four-phase roadmap to strong end-to-end security and identity. Since then,
we have deployed the first phase of the roadmap and expanded the use of end-to-end encryption and
other advanced cryptography to more Zoom products, such as Phone and Email. To better describe these
new developments, we moved away from a chronological presentation in favor of an evergreen document
that describes the cryptographic design behind each product.
