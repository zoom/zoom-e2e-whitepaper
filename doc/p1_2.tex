\subsubsection{Key Rotation}

At any point later in the meeting, the leader can generate a new 32-byte value $\mk'$. The leader performs steps \ref{participantJoinRekeyStart}-\ref{participantJoinRekeyEnd} of ``Participant Join (Leader)'' for all participants, with the updated $\mk'$ value. All participants see the rekey signal on their signaling channel, and perform steps \ref{participantJoinRekeyNonLeaderStart}-\ref{participantJoinRekeyNonLeaderEnd} of ``Participant Join (Non-Loader)''.  Participants should not immediately encrypt using the new meeting key; they should wait about 15 seconds, to ensure all participants smoothly transition over.

The leader should trigger a rekey whenever a participant enters or leaves the meeting, but not more than every 15 seconds (to prevent thrashing). As an important security measure, users joining the meeting never get to see keys that are more than 15 seconds old, since otherwise they could view encrypted in-meeting chats deep into the meeting's past. But likewise this rekeying strategy shouldn't delay a user from joining a meeting.

It's important to note each $\mk$ is independently generated, so knowing the previous $\mk$ provides no information about the subsequent $\mk'$.

Each $\mk$ has an associated sequence number $\mkSequenceNumber$, starting at 0 and incrementing whenever it rolls over. The leader sends this sequence number along with the participant list heartbeat below.

\subsubsection{Authenticated Participant Lists}

A meeting leader maintains a ``leader participant list'' ($\LPL$), tabulating
all users they have added to and removed from the meeting. Every time this list changes, the leader increments the counter v and posts the mapping $\langle v \rightarrow \LPL_v \rangle$ to the bulletin board. They broadcast a signature over this list whenever membership changes, and at designated ``heartbeat intervals'':
%
\begin{multline*}
\mathsf{DualSign}(\isk_{\LL}, (\texttt{"Zoom00LPL\textbackslash0"} \| \binding_{\LL} \| \mathsf{SHA256}(\LPL_v) \| v \| t \| \mkSequenceNumber ))
\end{multline*}
%
Where $t$ increments on every send, $v$ increments whenever the $\LPL$ changes, and $\mkSequenceNumber$ increments on every $\mk$ rotation.

The other participants should know this list, so they know who to rekey for if the leader drops out and they become the new leader. Evil servers might try to withhold updates the leader makes here, to hide when bad actors are kicked out. As such, the leader also sends a low bandwidth ``heartbeat'' over the control channel. Heartbeats should go out at least every 10 seconds. All participants observe and verify these heartbeats, and if they fail to receive four heartbeats in a row, they should drop out of the meeting.


\subsubsection{Meeting Teardown}
At the end of the meeting, or when leaving a meeting early, all participants should discard all meeting keys, all keys derived from those meeting keys, and the ephemeral DH private keys $\sk_i$ they generated when they joined.

The intent here is to provide forward secrecy. That is, if an adversary can record all encrypted messages relayed between Zoom clients during the meeting, and can later recover all keys stored on a user's device after the meeting ends, they still cannot recover the meeting data.

\subsubsection{Meeting Replay Attack}

What happens if Alice, Bob and Charlie meet at meeting m two days in a row, and a malicious Zoom server mixes bulletin board messages from the two meetings? In particular, what if Alice is the leader and the server presents Bob's previous ephemeral key? Alice doesn't realize it's Bob's previous key, so she'll obediently encrypt for it in step 4 of ``Participant Join (Leader)'' using her new ephemeral key. This will result in a message that Bob can't decrypt, since he'll be using his new ephemeral private key, having throw his old ephemeral private key away at the end of the last meeting. The malicious server can't roll back the whole conversation, since Alice, Bob and Charlie know to use their new ephemeral keys; they couldn't even use their old ones if they tried.

\subsection{Meeting Security Code}
If all participants can verify the authenticity of the leader's public key ($\ivk_{\LL}$), they are safe from ``meddler-in-the-middle attacks'' (MITM)\footnote{``Meddler-in-the-middle attack'' is also known as ``man-in-the-middle attack''.}. The Zoom client exposes the following ``meeting security code'' in the security tab:

$$\mathsf{Wordify}(\mathsf{Hash128}(\texttt{"Zoom00MSecCode\textbackslash0"} || \ivk_{\LL}))$$

$\mathsf{Hash128}$ is a standard hash function like SHA-256, but truncated to 128 bits. The $\mathsf{Wordify}$ function converts a binary hash to a human-readable and internationalization-friendly code. Crucially, every Zoom client in the meeting independently computes these values on the $\ivk_{\LL}$ used in the handshake protocol. The length of the code is long enough to protect against second pre-image attacks. The leader reads out the meeting security code, and everyone in the meeting in turn does the same thing. If the code does not match, the participant should speak up in the meeting, and the leader should rotate the meeting key by kicking them out; they may be allowed to rejoin and try again.

If deep fake technology is a concern, or the participants do not know each other in advance, this verification can also happen over a different out-of-band secure channel.

We considered other approaches to the meeting security code, such as mixing more of the handshake data into the displayed code. While more mixins would be more robust to attacks that try to confuse participants by mixing members from different meetings, we see a UX advantage of ``one leader, one code.''

\subsection{Changes to symmetric encryption}
Symmetric encryption in Zoom meetings will use AES-GCM with unique per-stream keys. As in the current design, all keys will be derived using a secure key derivation function (KDF) from a per-meeting Meeting Key ($\MK$). The meeting leader may rotate $\MK$ throughout the meeting.

All encrypted UDP packets are prefaced with the bottom 8 bits of $\mkSequenceNumber$, so participants know which version of $\MK$ to decrypt with. Wrapping is not a security concern, since users can trust the participant list heartbeats for the full $\mkSequenceNumber$.

\subsection{Abuse Management and Reporting}
If a user experiences abusive behavior and wishes to report it to Zoom's Trust and Safety team, they simply upload the unencrypted data normally collected in an abuse report (e.g. a description of the abuse and some portion of the meeting content) to Zoom for review. This protocol is imperfect, since it potentially could allow a bad meeting participant to ``frame'' an honest meeting participant for abuse that didn't happen. For the same reason, it allows an actual abuser to disavow uploaded evidence of their abuse. We think for now, the framing behavior is rare and only possible with access to good ``deep fake'' technology.

Future refinements are possible. Participants could sign their outgoing video streams, and other participants will only allow meetings to proceed if all streams are appropriately signed. This change would defeat the two attacks above, but with major drawbacks:
%
\begin{description}
    \item {\bf Performance:} Signing and verifying individual UDP video streams is expensive in terms of bandwidth and computation. More research is required to make this change practical.
    \item {\bf Repudiation:} Honest participants might not want an indelible record that they said something. They might understandably want to treat meetings as ephemeral in accordance with their standard data retention practices.
\end{description}

Given the challenges in this space, we will revisit these decisions at a later date as we gain more operational experience with the current proposal.

\subsection{Areas to Improve in Phase I}
Phase I is geared toward a quick deployment and providing building blocks for future development phases. As such, there are some potential attacks we do not fully cover here:
\begin{description}
\item {\bf Meddler-in-the-Middle.} The ``security code'' proposal above is a countermeasure for the classic ``meddler in the middle attack'', wherein Bob isn't actually connecting to Zoom, he's connecting to Eve who is proxying his communications. But our solution isn't perfect. First off, it can be defeated by deep fake technology; second, it's clunky from a UX perspective; and third, users joining the meeting late must request the ceremony be reperformed. We could work toward addressing all of these concerns, but future phases of development provide better solutions by building strong notions of identity.
\item {\bf Anonymous Eavesdropper.} An adversary, in conjunction with a malicious Zoom server, types in a name of their choosing, turns off video, mutes their microphone and just observes. We'll fix this problem with better identity guarantees in Phase II.
\item {\bf Impersonation Attacks Within the Meeting.} Even if Alice and Bob are both authorized to be in the meeting, if Alice has the help of a malicious server, she can inject audio/video for Bob. Charlie would have no way of knowing that Bob's stream was being faked.
\end{description}

The upshot here is that in Phase II, we look to further strengthen meetings through better identity.